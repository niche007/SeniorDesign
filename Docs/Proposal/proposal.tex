\documentclass[11pt]{article}
\usepackage{geometry}  
%\usepackage[margin=1cm]{geometry}
\usepackage{pgfgantt}
\usepackage{graphicx}
\usepackage{xcolor}
%\usetikzlibrary{positioning}

\ganttset{group/.append style={orange},
milestone/.append style={red},
progress label node anchor/.append style={text=red}}    



% See geometry.pdf to learn the layout options. There are lots.
\geometry{letterpaper}                   % ... or a4paper or a5paper or ... 
%\geometry{landscape}                % Activate for for rotated page geometry
%\usepackage[parfill]{parskip}    % Activate to begin paragraphs with an empty line rather than an indent
\usepackage{graphicx}
\usepackage[colorinlistoftodos]{todonotes}
\usepackage{amssymb}
\usepackage{epstopdf}
\usepackage[english]{babel}
\usepackage{placeins}
\usepackage{tikz}
\usepackage{pgf}
\usepackage{inputenc}
\usetikzlibrary{shapes,arrows}
\usetikzlibrary{automata,positioning}     
\usepackage{listings}
\usepackage{smartdiagram}
\usepackage{tocstyle}
\usetocstyle{standard}
\lstset
{ %Formatting for code in appendix
    language = C,
    basicstyle=\footnotesize,
}   

\tikzset{
    state/.style={
           rectangle,
           rounded corners,
           draw=black, very thick,
           minimum height=2em,
           inner sep=2pt,
           text centered,
           },
}
\DeclareGraphicsRule{.tif}{png}{.png}{`convert #1 `dirname #1`/`basename #1 .tif`.png}
\begin{document}
\begin{titlepage}

\newcommand{\HRule}{\rule{\linewidth}{0.5mm}} % Defines a new command for the horizontal lines, change thickness here

\center % Center everything on the page
 
%----------------------------------------------------------------------------------------
%	HEADING SECTIONS
%----------------------------------------------------------------------------------------

\textsc{\LARGE George Mason University}\\[1.5cm] % Name of your university/college


%----------------------------------------------------------------------------------------
%	TITLE SECTION
%----------------------------------------------------------------------------------------

\HRule \\[0.4cm]
{ \huge \bfseries Blind Deconvolution}\\[0.4cm] % Title of your document
\LARGE{\textbf{Proposal Document}}
\HRule \\[1.5cm]
 
%----------------------------------------------------------------------------------------
%	AUTHOR SECTION
%----------------------------------------------------------------------------------------
\begin{minipage}{0.4\textwidth}
\begin{flushleft} \large
\emph{Authors:}\\
Adrienne \textsc{Stotle} \\
Aneesh \textsc{Malhotra} \\ 
Rishi \textsc{Gupta} \\ 
Shakib \textsc{Rashid} % Your name 
\end{flushleft}
\end{minipage}
~
\begin{minipage}{0.4\textwidth}
\begin{flushright} \large
\emph{Supervisor:} \\
Dr. Yariv \textsc{Ephraim} % Supervisor's Name
\end{flushright}
\end{minipage}\\[2cm]

% If you don't want a supervisor, uncomment the two lines below and remove the section above
%\Large \emph{Author:}\\
%John \textsc{Smith}\\[3cm] % Your name

%----------------------------------------------------------------------------------------
%	DATE SECTION
%----------------------------------------------------------------------------------------

{\large \today}\\[2cm] % Date, change the \today to a set date if you want to be precise

%----------------------------------------------------------------------------------------
%	LOGO SECTION
%----------------------------------------------------------------------------------------

\includegraphics{watermark2.jpg}\\[1cm] % Include a department/university logo - this will require the graphicx package
 
%----------------------------------------------------------------------------------------

\vfill % Fill the rest of the page with whitespace

\end{titlepage}
\clearpage
\tableofcontents
\clearpage


Blind Deconvolution Group:
Aneesh Malhotra, Shakib Rashid, Adrienne Stotle, Rishi Gupta
\section{Summary of Project Requirements}

\subsection{Mission Requirements}

\begin{itemize}
\item The device will be code written in Matlab that will undo the convolution of an image $f(x,y)$ with a blurring impulse function $h(x,y)$

\end{itemize}

\subsection{Operational Requirements}


\begin{itemize}
\item The input to the code will be an image, which will be represented as an $m \times n \times 3$ array, which will account for the spatial components as well RGB channels of the image.  

\item The output will be another image that approximates the "true" image before the blurring function, as well as the statistical features of the Bayesian method that was used. 

The goal of our project is to replicate the results found in \cite{R01} in order to reconstruct an image that has been blurred using a known blurring function. The blurred image will be represented as the 2D convolution of the true image, $f(x,y)$, with a blurring function $h(x,y)$.
\item The original image will be obtained when the blurring function is a Guassian. 
\item The output will be some code written in Matlab that implements this method, and can be run on a standard laptop. 
\end{itemize}
\bibliographystyle{ieeetr}
\bibliography{proposal_refs}

\end{document}  